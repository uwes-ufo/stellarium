% Status info:
% 2022-09-19 Foreword for V1.0

\chapter*{Foreword for Stellarium 1.0}
\label{ch:Foreword}

\section*{Version 1.0?}

More than twenty years have elapsed since Fabien Ch\'ereau typed the
first lines of code of a desktop planetarium which should be based on
modern algorithms from computer graphics to provide a trustworthy
simulation of the night sky, utilizing OpenGL technology which best
runs on dedicated graphics hardware. What evolved from these first
attempts has evolved into a beautiful program which has been
downloaded millions of times to run on desktop computers and notebooks
all around the planet, on various operating systems and in dozens of
languages.

Already around 2010 the first team of developers started to discuss
when the ``zero'' from the version number should be dropped. This step
is important in software development: the zero version signals a
program is not finished yet, may be unstable, may deliver wrong
results or may have other deficiencies. My stance for a long time was
to keep the ``zero'' until Stellarium's astronomical accuracy would
allow its use for historical research without headaches or
warnings. Stellarium should be a system that at least works as good as
other systems deemed ``reliable''. However, for a long time I always
had to warn my peers from the historical domains that results should
be cross-checked with ``better'' tools. Cross-checking still is
recommended, but it has become hard to find tools that were better and
not just based on the same best physics-based models which we have
been starting to use in recent years. Appendix~\ref{ch:Accuracy}
provides some more details into this topic.

In the first phase of development until around 2012, Stellarium's look
and feel were shaped. Since the time I first took notice of the
program, around 2008, it has featured many functionalities for amateur
astronomers who want to know what to expect when they leave their
house. Some user reactions from this time praised its completeness in
observer-specific features, e.g.\ the Telescope Control and Oculars
plugins, but also its ease of use. Its Satellites plugin was capable of
simulating even Iridium flares, which are however no longer visible as
the satellites have been de-orbited.

After the switch to the Qt5 programming foundation for the 0.13
release, most of the first developer team, whose work still forms the
core of the program, retreated, leaving maintenance and further
development mostly to the two of us, a computer scientist researching
in the domain of cultural astronomy, and another whose daytime job is
high school physics teacher. This somewhat influenced further
development. For simulation of historical sky views, more accurate
astronomical models had to be implemented. I admit, finding my way
through the code written by others was not easy, and despite some
institutional support we also cannot invest too much of our time into
software development. Still, my work to improve the astronomical
features of Stellarium concentrated on replacing, figuratively
speaking, ``sprockets'' by full-grown ``gears'' described in the
current scientific literature, until, finally, in mid-2021 we had a
program that fulfilled my demands. Time to drop the zero? Almost!

For more than a decade the Qt programming framework has been the
technical foundation of Stellarium which allows its deployment on
multiple platforms. Especially the Qt5 framework, which unified all
graphics output utilizing the OpenGL graphics technology, allowed
deployment on Windows, Mac and Linux systems ranging from 2008
Windows~XP PCs to the tiny Raspberry Pi~3 or 4 platforms. Windows~7/10
PCs with insufficient graphics hardware could utilize Google's ANGLE
library which translated OpenGL ES2.0 (a limited subset of OpenGL
which does everything Stellarium needs) to DirectX. Almost all desktop
computer systems were capable of running Stellarium!

The Qt project has released its next generation of the framework, Qt6,
in late 2020. Stellarium has to follow Qt's development quite closely,
and so we rapidly decided that ``Stellarium 1.0'' still had to wait
until we upgraded the software onto the new Qt6.

However, also computer platforms evolve. OpenGL has fallen out of
favour for Apple and Microsoft who favour their own graphics
libraries, and Vulkan has been introduced as modernized technical
successor of OpenGL. Therefore, the ANGLE library on Windows is no
longer supported with Qt6, and future developments may move away from
OpenGL. For the time being, we continue to use it, though.

Over the past year, we put most effort into this upgrade. Finally, in
summer of 2022, Stellarium is ready for Qt6. But: is your computer
ready for it?

Qt6 has left some computers behind: it requires 64-bit
processors. Windows~10 is seen as minimum version of this operating
system. This poleaxes all those of us who try to keep running
e.g.\ observatory gear with somewhat older or outdated systems or
simply cannot afford the latest hardware.

What can we do about this?


The last purely Qt5-based version was version 0.22.2 released at June
solstice 2022. For several years now, the number after the zero has
indicated the year of release, and many users have already dropped the
zero mentally when discussing versions.  The logical new number for
the autum 2022 release would have been 0.22.3.

In fact, we can still produce Qt5-based builds with the same
functionality from the same source code. But we wanted to base ``1.0''
on Qt6, right? Yes, and we still do, but not as we had hoped. We
cannot leave behind Qt5 so fast!


Instead, we decided the following: We keep the internal series
number 0 for Qt5-based builds, and use series 1 for Qt6-based
builds. This is the number which is part of the downloaded
installation package.  But is version 1.22.3 ``better'' than 0.22.3?
Not necessarily. It is the more modern, sure, but it provides the same
astronomical results. It requires more modern hardware (64-bit and
esp.\ full OpenGL driver support on Windows!), and has tiny
differences in the scripting capabilities (see
chapter~\ref{ch:scripting}). However, in terms of features, both could
be called ``Version 22.3''. Just like that? No, sorry.
We really want to mark this release, which marks both ``accurate'' and
``future-proof'', as something special. Therefore we give this release
the version number 1.0 to mark ``\emph{Stellarium is finished}''.

Of course, software is never finished. The next release, probably due
at December solstice 2022, will be called version 1.1, but then we
will go back to year-based numbers like 23.0, 23.1, etc. We will see
from our download counters how much demand will there be for future
Qt5-based builds and when it will be time to drop the 0 series. 


\section*{What changed visibly in version 1.0?}

Just as we prepared for this release, our contributor Ruslan
Kabatsayev came forward and said his development of a new skylight
model has come to a point where it is ready for deployment. I had
followed this development for years with great interest as it really
provides a stunning simlation of twilight colors. So, apart from minor
improvements elsewhere, the most notable change you will be able to
enjoy is this new skylight model. Unfortunately though, for technical
reasons, this mode is not available for Apple computers with their
peculiar OpenGL implementation.  See chapter~\ref{ch:skylight} for
details.

Another notable recent development are contributions of Worachate
Boonplod wo concentrated on eclipse predictions in the AstroCalc
module (see section~\ref{sec:gui:AstroCalc:Eclipses}).  This also
combines superbly with the new eclipse sky simulation!

\vspace{2\baselineskip}



In the name of all prior developers we wish you much enjoyment with
this and future versions of the Stellarium desktop planetarium!


\vspace{2\baselineskip}

Georg Zotti in September 2022.

\vspace{2\baselineskip}

% TODO: Greetings/notes from Alexander Wolf
% Alexander Wolf in September 2022.


%%% Local Variables: 
%%% mode: latex
%%% TeX-PDF-mode: t
%%% TeX-master: "guide"
%%% End: 
